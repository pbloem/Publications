\documentclass{article}

\usepackage{charter}
\usepackage{eulervm}
\usepackage{amsmath, amsthm, amssymb}

\theoremstyle{definition}
\newtheorem*{thm}{Theorem}
\newtheorem{lma}{Lemma}
\newtheorem*{dfn}{Definition}
\newtheorem*{exm}{Example}

\title{Measuring Dimension}
\date{\today}

% non-indented, spaced paragraphs
\setlength{\parindent}{0.0in}
\setlength{\parskip}{0.1in}

\begin{document}

\maketitle

\begin{abstract}
The intrinsic dimension of a dataset is one of the most telling statistics that can be calculated, and one that has received a great deal of attentation over the past decades. Most methods for estimating dimension relsy on at least some subjective judgment to make a selection in the data. Adapting methods recently described [@! cite] for the statistical analysis of power laws, we revisit an existing MLE estimator and present a simple, objective procedure to estimate the dimension of a dataset. The procedure runs without human intervention, and produces a value for the dimension, uncertainty bounds and a significance value.\end{abstract}

\section{Introduction}

Definitions of dimension, fractal dimension and measuring dimension.

Previous work. Theiler. Manifold learning. Databases (Cite faloutsos)We will not discuss the mathematical backgrounds of dimension (Cite Edgar). We will not focus greatly on the algorithmic complexity.

\section{Dimension}

Scaling laws.

Box counting dimension, Correlation integral, Haussdorff & Packing dimension.

Box counting estimator, Correlation estimator, Takens estimator.

\subsection{Physics}
\subsection{Statistics and Machine Learning}
\subsection{}

\section{Methods}

\subsection{Maximum Likelihood Estimator}

Derivation of Takens' MLE and its standard error. Comparison to the Hill estimator and discussion of the differences.
 
\subsection{Estimating the $d_Max$ parameter}

Quick review of the methods described in Clauset. Translation of the method to this estimator. Experiment showing the validity of the method for this MLE.

\subsection{Uncertainty}

Quick review of the approach of Clauset.

\subsection{Significance}

Review of the approach of Clauset, and a description of the method translated to this problem.

\subsection{Metric data}

Discussion of the use of dimension for metric data and the use of the significance measure.

\subsection{Runnning time}

Discussion of the running time an an investigation of the effects of reducing the data size.
\section{Experiments}
\subsection{Data sets}

\begin{itemize}
  \item Swiss roll
  \item 8-loop
   
  \item Hands
  \item Faces

  \item Sierpinski
  \item Cantor
  \item Menger Sponge
   
  \item 
\end{itemize}

\subsection{Results}

\section{Conclusion}

We have presented a straightforward, objective method for calculating a maximum likelihood estimate of the intrinsic dimension of a dataset. We have com 

\end{document}
