\documentclass{thesis}

%\usepackage{amsthm}
%\usepackage{charter}
%\usepackage{eulervm}
\usepackage{amsmath}
\usepackage{amssymb}
\usepackage{graphicx}
\usepackage{caption}
\usepackage{subcaption}
\usepackage{enumerate}
\usepackage{cancel}
\usepackage{hyperref}

%\theoremstyle{definition}
\newtheorem{thm}{Theorem}
\newtheorem{crl}{Corollary}
\newtheorem{lma}{Lemma}
\newtheorem{dfn}{Definition}
\newtheorem{exm}{Example}

\newenvironment{out}
 {\emph{Proof outline.\;}}
 {}

%\let\doendproof\endproof
%\renewcommand\endproof{~\hfill\qed\doendproof}

\usepackage{color}
\usepackage[usenames,dvipsnames,svgnames,table]{xcolor}

\DeclareMathOperator*{\argmin}{\arg\,\min}
\DeclareMathOperator*{\argmax}{\arg\,\max}
\DeclareMathOperator*{\nid}{NID}
\DeclareMathOperator*{\id}{ID}


\newcommand{\sdr}[1]{\textcolor{blue}{\small #1\textsuperscript{[sdr]} }}
\newcommand{\pb}[1]{\textcolor{OliveGreen}{\small #1 \textsuperscript{[pb]} }}

\newcommand{\p}{\mbox{\,.}}
\newcommand{\fl}[1]{\left \lfloor #1 \right \rfloor}
\newcommand{\g}[1]{\color{gray} #1 \color{black}}
\newcommand{\B}{{\mathbb B}}
\newcommand{\R}{{\mathbb R}}
\newcommand{\N}{{\mathbb N}}
\newcommand{\m}{{\overline{m}}}
\newcommand{\ok}{{\overline{\kappa}}}
\newcommand{\mmid}{\;\middle|\;}

\newcommand{\hide}[1]{}

\title{Statistics for complex graphs: an algorithmic approach}
\date{\today}

% non-indented, spaced paragraphs
\setlength{\parindent}{0.0in}
\setlength{\parskip}{0.1in}

\begin{document}

\chapter{Introduction}

% Start with a story
% Mention RDF: the future of data storage 

\section{The complex graph}

% Complex graphs are not just funny things to look at, the are the fundamental unit of science in the 21st century. 

\section{Computer programs as statistical models}

\chapter{Statistical analyses of complex graphs}

\begin{summary}We review the literature on complex graphs, and the approaches that have been used so far to provide statistical analyses on them.
\end{summary}

% Algorithmic statistics is how you stud graphs. Nothing else makes sense.

\chapter{Algorithmic Statistics: Bridging the gap between Kolmogorov complexity and MDL}

\begin{summary}MDL and Kolmogorov complexity are two offshoots of the same basic idea: that good compression makes good statistics. MDL takes this as an axiom and builds on it, while Kolmogorov complexity makes only the assumption that the data is produced by a computational process. The two methods are not incompatible, but they have produced very different results and techniques. In order to have access to the best of both worlds, we start with the rigorous theoretical foundations of Kolmogorov complexity and build a platform that will allow us to fold in the techniques of MDL. We will revisit the principle of two-part coding, considered to be a usable, but cumbersome technique in MDL and show that when combined with Kolmogorov complexity it has some deep advantages over other methods.
\end{summary}

\chapter{What makes data interesting: two-part coding in action}

\begin{summary}The question of what makes data complex is has been decisively answered. First for IID data in, as maximal Entropy, and then for any data as maximal Kolmogorov Complexity. But there's some dissonance in calling this data information-rich. We are not very interested in watching TV broadcasts of static, or recordings of white noise. This may contain maximal information to a computer, but to us, it's not very interesting. In fact the most interesting data seems to be somewhere in between the zero complexity signal (s tv which is turned off) and the maximum complexity one (analog noise). The question of whether we can capture this notion of `interestingness' in a metric analogous to information entropy and Kolmogorov complexity has occupied research for over half a century, and has produced some surprisingly deep results. This chapter provides a review.
\end{summary}

\section{The question of interestingness}

\chapter{Graph modeling: extracting motifs}

\begin{summary}Modeling complex objects starts with finding out what the components are. If we know the letters, we can find the words. If we know the words, we can find the sentences. Graphs give us nodes and links for free. How do they combine into functional, recurring units? How can we build up to the global structure?
\end{summary}

\chapter{Graph modeling with rewrite grammars}

\begin{summary}In last chapter we induced motifs for various types of complex networks. We ended by applying the procedure recursively, collapsing upwards across several levels to compress the whole structure.
\end{summary}

\chapter{Graph modeling with deep neural networks}

\begin{summary}To model graphs in an algorithmic setting we can use any family of computational structures that can turn randomness into graph structures. One recently popular model that has been shown to be effective at producing random examples of any given model is the Deep Belief Network. We investigate their potential to model graph structure, and fit them in to the framework of algorithmic statistics.
\end{summary}

\chapter{Machine learning on graphs}

% This does not fit the line too well, refocus?

\begin{summary}The classic view of offline machine learning starts with a list of instances from which to learn. These instances can then be represented in some way---preferably a vector of numeric features---and analyzed statistically. In the past decade a situation has emerged and re-emerged that is an ill fit to this paradigm. We are faced with information in a graph: a social network, a web graph, an RDF dataset. We may know our instances, if they correspond to specific nodes in the graph, but the things we know about them are spread throughout the graph. We investigate the problem and provide a bridge to traditional machine learning.
\end{summary}

\section{A pipeline}
\subsection{Instance extraction}
\subsection{Feature extraction}
\section{Methods}
\subsection{Classification}
\subsection{Clustering}
\subsection{Regression}

\chapter{Citation networks, a use case}

\begin{summary}Throughout the previous chapters, we have performed many quick tests on small datasets, to illustrate and validate our points. In this chapter we take one use case and investigate more thoroughly, in order to show the full potential of our methods. We choose the domain of citation networks, and investigate various questions.
\end{summary}

\begin{itemize}
  \item Who are the hubs?
  \item H-index: can we do better?
  \item Multi-resolution analysis
  \item Visualization
\end{itemize}

\chapter{Statistics for complex graphs}

\begin{summary}\ldots
\end{summary}

Conclusatory chapter.

\appendix
\chapter{Tutorials}
\section{Kolmogorov Complexity}
\section{Minimum Description Length}
\section{Graphs and RDF}
\chapter{Algorithms and techniques}
\chapter{Proofs}

\end{document}
