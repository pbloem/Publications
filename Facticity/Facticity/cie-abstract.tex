\documentclass{style/llncs}

\usepackage{amsmath,amsfonts}
\usepackage{color}
\usepackage[usenames,dvipsnames,svgnames,table]{xcolor}
\usepackage[mathscr]{eucal}
\usepackage{thmtools}
\usepackage{graphicx}
\usepackage{caption}

\newcommand{\M}{\mathscr M}
\newcommand{\C}{\mathscr C}
\newcommand{\T}{\mathscr T}
\newcommand{\F}{\mathscr F}
\renewcommand{\P}{\mathscr P}
\newcommand{\K}{\mathscr K}
\newcommand{\X}{\mathscr X}
\newcommand{\B}{\mathbb B}
\newcommand{\D}{\Delta}
\newcommand{\N}{\mathbb N}
\newcommand{\tn}[1]{\textnormal{#1}}
\newcommand{\pair}[1]{\left\langle{#1}\right\rangle}
\newcommand{\concat}{\oplus}
\newcommand{\symb}[1]{\texttt{#1}}
\newcommand{\br}[1]{\overline{#1}}
\newcommand{\s}{S}
\newcommand{\dom}[1]{\mathop{\tn{dom}(#1)}}
\newcommand{\range}[1]{\mathop{\tn{range}(#1)}}

\newtheorem{conj}{Conjecture}

\let\doendproof\endproof
\renewcommand\endproof{~\hfill\qed\doendproof}

\newcommand{\p}{\,\text{.}}

\newcommand{\tuple}[1]{\left\langle{#1}\right\rangle}

\newcommand{\hide}[1]{}
\newcommand{\old}[1]{}

\newcommand{\sdr}[1]{\textcolor{blue}{\small #1\textsuperscript{[Steven]} }}
\newcommand{\pb}[1]{\textcolor{OliveGreen}{\small #1 \textsuperscript{[Peter]} }}

\newcommand{\argmin}{\mathop{\arg\min}}

% --- DELETE BEFORE SUBMISSIONS ---
% \pagestyle{headings} 


\title{The problem of Sophistication}

\author{Peter Bloem and Steven de Rooij}

\institute{
  System and Network Engineering Group, \\University of Amsterdam, the Netherlands\\
  \email{uva@peterbloem.nl, steven.de.rooij@gmail.com}
}

\begin{document} 
\maketitle

\begin{abstract}
Kolmogorov complexity is a sound, unambiguous formalization of the information content of an object. However, researchers have long been plagued by its inability to express a more intuitive notion of complexity: in terms of Kolmogorov complexity, a completely random bitstring is the most complex object possible. This is not what we mean when we we use the word complexity informally: a television broadcast is complex when we can see structures, faces and landscapes, not when we can only see white noise. 

\hspace{0.05\textwidth} The desire to capture this notion of complexity in formal terms has led to many proposals for additional measures. The largest family of proposals is built on the following idea:  an object consists of \emph{structural} and \emph{stochastic} information, and the information content of the structural information determines the complexity of the object. We will use the name \emph{sophistication} for such proposals. 

\hspace{0.05\textwidth} In essence, sophistication performs model selection using the class of all computable models. This means that we are dealing not just with the problem of formally defining complexity, but also with a limiting case of statistics: what happens if we extend our model class to the entirety of conceivable models? Do we end up with an ultimate statistics machine, or does the whole framework collapse, and does model selection only work with carefully constrained model classes? 

\hspace{0.05\textwidth} While sophistication is a simple enough idea, any attempt to formalize it quickly runs into problems. We highlight the two major issues, and show how they invalidate each proposal. We are forced to conclude that no current proposal truly solves the problem of sophistication, and we offer some informal arguments which suggest that no such formalization may be possible at all.

\end{abstract}

\end{document}