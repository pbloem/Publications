\documentclass{article}

\usepackage{charter}
\usepackage{eulervm}
\usepackage{amsmath, amsthm, amssymb}

\theoremstyle{definition}
\newtheorem*{thm}{Theorem}
\newtheorem{lma}{Lemma}
\newtheorem*{dfn}{Definition}
\newtheorem*{exm}{Example}

\title{Fractal Modeling}
\date{\today}

% non-indented, spaced paragraphs
\setlength{\parindent}{0.0in}
\setlength{\parskip}{0.1in}

\begin{document}

\maketitle

\begin{abstract}
Fractals are mathematical objects that display great detail and complexity from a simple description. It has often been suggested that fractals mimic the structures of nature and everyday life better than the shapes of classical geometry \cite{}. These properties make fractals interesting candidates for modeling data. We present an overview of existing concepts that are relevant to this task, and a new algorithm for fitting fractal models to data. 
\end{abstract}


\section{Preliminaries}
\subsection{Fractals}

Fractals are mathematical objects (figures, sets or measures) with certain remarkable properties. No official definition exists for what does and does not constitute a fractal, but the following properties are central to all treatments:
\begin{itemize}
  \item They are self-similar. A part is a scaled down copy of the whole. See for instance the Sierpinski triangle (fig !@), which can be seen as three scaled down copies of itself. In some cases the self similarity is statistical, so that the scaled down copy is statistically similar to the whole, but not exactly so.
  \item They have infinite detail. 'Zooming in' reveals ever finer detail. In the case of the Sierpinski triangle we will simply see the same shape recurring again and again, but fractals like the Mandelbrot set reveal a great variety of detail.
  \item They can have non-integer dimension, for instance the Sierpinski triangle has dimension about 1.58. This basic and counter-intuitive fact is not easily explained. It will be discussed in section @!.
\end{itemize}

The earliest examples of fractals were described around the beginning of the twentieth century as counterexamples to conjectures in planar geometry and mostly considered pathological. It was not until the 1970s that they began to be seen as a single and useful family of mathematical objects that could accurately describe many physical phenomena. [@! Cite mandelbrot] Beno\^it Mandelbrot, commonly regarded as the father of fractal geometry, put it as follows in the \emph{The Fractal Geometry of Nature} [@! cite]:

\begin{quotation}
\noindent Clouds are not spheres, mountains are not cones, coastlines are not circles, and bark is not smooth, nor does lightning travel in a straight line.
\end{quotation}

\subsection{Iterated Function Systems}


\section{Methods}

\subsection{Dimension}

\section{Modeling}

There are many well-known algorithms for 

One of the main challenges in all forms of fractal modelling has been \textit{the inverse problem}. Whereas the chaos game produces a set of points from a fractal model, the inverse problem is to find a fractal for a given set of data points. In short, fitting a fractal model to data.

So far, two approaches have succesfully tackled the inverse problem in some form:
\begin{itemize}
  \item Genetic algorithms!@. A straightforward search through the space of IFSs. The drawback of this approach is one of time complexity. The running time of these approaches scale poorly in the number of dimensions and components.
  \item Partitioned iterated function systems!@. This is a modified version of the problem to make the model tractable and feasible in the domain of image compression. This approach does not translate well to n-dimensional point patterns.
\end{itemize}


\subsection{Classification}
\section{Conclusions and Further Work}

One advantage of the IFS model that has been left unexplored in the current work is that it assigns each point in the dataset a $k$-ary string so that a small distance between points suggest simlar strings. Thus, a point pattern in a Euclidean space is naturally mapped to the space of strings in a way that uses the structure of the dataset rather than the structure of the embedding space.

Future investigation might focus on the use of IFS models in clustering, spatial indexing and the induction of random fractals (as discussed in \cite{})
\end{document}