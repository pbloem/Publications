%%This is a very basic article template.
%%There is just one section and two subsections.
\documentclass{article}

\usepackage{charter}
\usepackage{eulervm}
\usepackage{amsmath, amsthm, amssymb}

\theoremstyle{definition}
\newtheorem{thm}{Theorem}
\newtheorem{lma}{Lemma}
\newtheorem*{dfn}{Definition}
\newtheorem*{exm}{Example}

\title{Kolmogorov Complexity}
\date{\today}

% non-indented, spaced paragraphs
\setlength{\parindent}{0.0in}
\setlength{\parskip}{0.1in}

\begin{document}

\maketitle

\begin{abstract}
\noindent This article aims to highlight some of the basic properties of the sawtooth diagram for facticity versus complexity.
\end{abstract}

\section*{The random strings}

As is well known, the majority of strings is random. Thus, their minimal model is that of all strings (ie. the copying Turing Machine). As n goes to infinity, the set has density 1. That means that if we create a density plot of an approximation of $K(x)$ versus $\phi(x)$ for all strings of a reasonable length, we will only see a single point in the bottom right of the diagram. 

If on the other hand, we do a scatter plot we are likely to see the rough structure of the facticity curve, but in the limit, the whole area below $K(x) = \phi(x)$ and $\phi(x) = \frac{1}{2}$ will be filled, obscuring most of the interesting detail.

To study the structure of this diagram in detail, we need to restrict ourselves to a subsets of strings. 

\section*{The simple Shannon models}

To start with, we can assume that we are studying compressible strings with an 
uneven number of ones and zeroes. For such strings, we have a simple compression scheme: store the frequency of ones in a prefix code and then code the rest of the string in $nH(x)$ bits.


\section*{Complex Shannon models}


\end{document}
