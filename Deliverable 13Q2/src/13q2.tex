\documentclass{article}

%\usepackage{amsthm}
\usepackage{charter}
\usepackage{eulervm}
\usepackage{amsmath}
\usepackage{amssymb}
\usepackage{graphicx}
\usepackage{caption}
\usepackage{subcaption}
\usepackage{enumerate}
\usepackage{cancel}
\usepackage{hyperref}

\usepackage{color}
\usepackage[usenames,dvipsnames,svgnames,table]{xcolor}

\DeclareMathOperator*{\argmin}{\arg\,\min}
\DeclareMathOperator*{\argmax}{\arg\,\max}

\newcommand{\sdr}[1]{\textcolor{blue}{\small #1\textsuperscript{[sdr]} }}
\newcommand{\pb}[1]{\textcolor{OliveGreen}{\small #1 \textsuperscript{[pb]} }}

\newcommand{\p}{\mbox{\,.}}
\newcommand{\fl}[1]{\left \lfloor #1 \right \rfloor}
\newcommand{\g}[1]{\color{gray} #1 \color{black}}
\newcommand{\hide}[1]{}

\author{Peter Bloem \and Pieter Adriaans}

\title{Empirical survey of complexity measures on graphs, focusing on RDF data and Algorithmic Statistics}
\date{\today} 

% non-indented, spaced paragraphs
\setlength{\parindent}{0.0in}
\setlength{\parskip}{0.1in}


\begin{document}


We have scaled up the results from the previous work package in data volume and added various complexity measures. We will process more datasets, and attempt to establish interesting empirical results based on compression and complexity analysis. The existing implementations of complexity measures from 2012Q3 are included as a comparison.

The purpose of this deliverable is threefold. It validates that our methods, such as they are, can be used to provide useful empirical results. It will serve as a test of our software and make sure that it is maintained (preventing 'code rust'). And finally, it will provide us with feedback for which features of our code are interesting enough to scale up in Q3.

\section{Datasets}

\section{Complexity measures}

\section{Results}

\section{Interpretation}

\end{document}
